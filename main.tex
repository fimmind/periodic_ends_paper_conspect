\documentclass[a4paper, 12pt]{article}
\usepackage[a4paper]{geometry}

\usepackage[utf8]{inputenc}
\usepackage[english]{babel}

\usepackage{amsmath}
\usepackage{amssymb}
\usepackage{amsthm}
\usepackage{mathrsfs}
\usepackage{mathtools}
\usepackage{booktabs}

\usepackage[
    colorlinks=true,
    allcolors=black,
    urlcolor=blue,
]{hyperref}

\newtheorem{theorem}{Theorem}[section]
\newtheorem{corollary}{Corollary}[theorem]
\newtheorem{lemma}[theorem]{Lemma}
\newtheorem{proposition}[theorem]{Proposition}

\theoremstyle{definition}
\newtheorem{definition}[theorem]{Definition}
\newtheorem{example}[theorem]{Example}

\begin{document}

\section{Motivation}

If \({ M }\) is a smooth closed orientable manifold of dimension \({ n }\), then the de Rham complex of complex-valued differential forms on \({ M }\), \[
    0 \longrightarrow \Omega^{0}(M) \overset{d_0}{\longrightarrow} \Omega^{1}(M) \overset{d_1}{\longrightarrow} \Omega^2(M) \longrightarrow \cdots \longrightarrow \Omega^{n}(M) \longrightarrow 0\,,
\] is known to be Fredholm in a suitable \({ L^{2} }\) completion. This means that the images of \({ d_{k} }\) are closed and the vector spaces \({ \ker d_{k} / \operatorname{im} d_{k-1} }\) are finite-dimensional. The alternating sum of the dimensions of these spaces is called the index of the de Rham complex. The de Rham theorem states that \({ \ker d_{k} / \operatorname{im} d_{k-1} \cong H^{k}(M; \mathbb C) }\), implying that the above index equals \({ \chi(M) }\), the Euler characteristic of \({ M }\).

The goal of the paper is to extend these results to manifolds periodic ends.

\section{Periodic Manifolds}

A manifold \({ M }\) with periodic end modeled on an infinite cyclic cover \({ \tilde X }\) of a compact manifold \({ X }\) associated with a primitive cohomology class \({ \gamma \in H^{1}(X;\mathbb Z) }\) is a Riemann manifold of the form \[
    Z_{\infty} = Z \cup W_0 \cup W_1 \cup W_2 \cup \cdots\,,
\] where \({ W_{k} }\) are isometric copies of the fundamental segment \({ W }\) obtained by cutting \({ X }\) open along an oriented connected submanifold \({ Y }\) and \({ Z }\) is a smooth compact manifold with boundary \({ Y }\). 

The completion of the de Rham complex of \({ M }\) in the \({ L^{2} }\) norm using over the end a Riemann measure \({ dx }\) lifted from that on \({ X }\) is not Fredholm. To rectify this, we will use \({ L_{\delta}^{^2} }\) norms, which are the \({ L^{2} }\) norms on \({ M }\) with respect to the measure \({ e^{\delta f(x)}\: dx }\) over the end. Here \({ \delta }\) is a real number and \({ f : \tilde X \to \mathbb R }\) is a smooth function such that \({ f(\tau(x)) = f(x) + 1 }\) with respect to the covering translation \({ \tau : \tilde X \to \tilde X }\). We shall denote the \({ L_{\delta}^2 }\) completion of the de Rham complex on \({ M }\) by \({ \Omega_{\delta}^*(M) }\).

\begin{theorem}
    \label{thm:Omega_delta_is_Fredholm}
    Let \({ M }\) be a smooth Riemannian manifold manifold with a periodic end modeled on \({ \tilde X }\), and suppose that \({ H_{*}(M; \mathbb C) }\) is finite-dimensional. Then \({ \Omega_{\delta}^*(M) }\) is Fredholm for all but finitely many \({ \delta }\) of the form \({ \delta = \ln \lvert \lambda \rvert }\), where \({ \lambda }\) is a root of the characteristic polynomial of \({ \tau_{*} : H_{*}(\tilde X; \mathbb C) \to H_{*}(\tilde X; \mathbb C) }\).
\end{theorem}

Given a manifold \({ M }\) as in the above theorem, the complex \({ \Omega_{\delta}^*(M) }\) has a well-defined index \({ \operatorname{ind}_{\delta}(M) }\). It is known, due to Miller, that \({ \operatorname{ind}_{\delta}(M) }\) is an even or odd function of \({ \delta }\) according to whether \({ \dim M = n }\) is even or odd, and that \({ \operatorname{ind}_{\delta}(M) = (-1)^{n} \chi(M) }\) for sufficiently large \({ \delta > 0 }\). The result of following theorem completes the calculation of the function \({ \operatorname{ind}_{\delta}(M) }\).

\begin{theorem}
    Let \({ M }\) be as in Theorem~\ref{thm:Omega_delta_is_Fredholm}. Then \({ \operatorname{ind}_{\delta}(M) }\) is a piecewise constant function of \({ \delta }\) whose only jumps occur at \({ \delta = \ln \lvert \lambda \rvert }\), where \({ \lambda }\) is a root of the char\-ac\-ter\-is\-tic polynomial \({ A_{k}(t) }\) of \({ \tau_{*} : H_{k}(\tilde X; \mathbb C) \to H_{k}(\tilde X; \mathbb C) }\) for some \({ k \in [0 : n-1] }\). Every such \({ \lambda }\) contributes \({ (-1)^{k+1} }\) times its multiplicity as a root of \({ A_{k}(t) }\) to the jump.
\end{theorem}

To be precise, we have a formula \[
    \operatorname{ind}_{\delta}(M) = (-1)^{n} \chi(M) + \sum (-1)^{k} \ \#\!\left\{ \lambda \mid A_{k}(\lambda) = 0\,, \lvert \lambda \rvert > e^{\delta} \right\}\,.
\]

\section{Finite Dimensionality}

Let \({ M }\) be a smooth orientable manifold with a periodic end modeled on \({ \tilde X }\). It is stated that vanishing of \({ \chi(X) }\) is a necessary yet not sufficient condition for the vector space \({ H_{*}(M; \mathbb C) }\) to be finite dimensional. To obtain a sufficient condition, observe that the derivative \({ df }\) defines a closed \({ 1 }\)-form on \({ X }\) and let \({ \xi = [df] \in H^{1}(X; \mathbb C) }\) be its cohomology class\footnote{Recall that \({ f : \tilde X \to \mathbb R }\) denotes a smooth function such that \({ f(\tau(x)) = f(x) + 1 }\) with respect to the covering translation \({ \tau : \tilde X \to \tilde X }\).}. The cup product with \({ \xi }\) gives rise to the chain complex
\begin{equation}
    \label{eq:the_cup_product_chain_complex}
    H^{0}(X; \mathbb C) \overset{\cup \xi}{\longrightarrow} H^{1}(X; \mathbb C) \overset{\cup \xi}{\longrightarrow} \cdots \overset{\cup \xi}{\longrightarrow} H^{n}(X; \mathbb C)\,.
\end{equation}

\begin{proposition}
    Suppose the chain complex~(\ref{eq:the_cup_product_chain_complex}) is exact. Then \({ H_{*}(M; \mathbb C) }\) is a finite-dimensional vector space for any smooth orientable manifold with periodic end modeled on \({ \tilde X }\).
\end{proposition}

\section{Examples}

\begin{example}
\hyphenation{man-i-fold}
A manifold with product end is a smooth Riemannian manifold whose end is modeled on \({ \tilde X = \mathbb R \times Y }\), where \({ Y }\) is a closed Riemannian manifold. The metric on \({ \mathbb R \times Y }\) is presumed to be the product metric. The covering translation induces an identity map \({ \tau_{*} }\) on the homology of \({ \mathbb R \times Y }\). Since \({ \lambda = 1 }\) is the only root of the characteristic polynomial of \({ \tau_{*} }\), the complex \({ \Omega_{\delta}^{*}(M) }\) is Fredholm for all \({ \delta \neq 0 }\). Its index \({ \operatorname{ind}_{\delta}(M) }\) equals \({ \chi(M) }\) if the dimension of \({ M }\) is even, and \({ -\operatorname{sgn}\delta \cdot \chi(M) }\) if the dimension of \({ M }\) is odd. Note that the same is true for any manifold whose periodic end is modeled on \({ \tilde X }\) such that the characteristic polynomial of \({ \tau_{*} : H_{*}(\tilde X; \mathbb C) \to H_{*}(\tilde X; \mathbb C) }\) only has unitary roots.
\end{example}


\end{document}
